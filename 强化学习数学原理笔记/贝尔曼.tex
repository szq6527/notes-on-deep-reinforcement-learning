%%%%%%%%%%%%%%%%%%%%%%%%%%%%%%%%%%%%%%%%%%%%%%%%%%%%%%%%%%%%%%%%%%%%%%%%%%%%%%%
%
% 您可以将其粘贴到Overleaf等在线LaTeX编辑器中直接编译。
%
% 主要使用的宏包 (Packages):
% - ctex: 提供中文支持,确保标题、文本中的中文能正确显示。
% - amsmath, amssymb: American Mathematical Society 提供的宏包,
%                      用于排版复杂的数学公式和符号。
% - geometry: 用于设置页面边距,使版面更美观。
% - enumitem: 用于优化列表环境(itemize, enumerate)。
% - hyperref: 生成PDF书签和可点击的链接,提升阅读体验。
%
%%%%%%%%%%%%%%%%%%%%%%%%%%%%%%%%%%%%%%%%%%%%%%%%%%%%%%%%%%%%%%%%%%%%%%%%%%%%%%%

\documentclass[a4paper]{article}

% --- 中文与排版设置 ---
\usepackage{ctex} % 中文支持
\usepackage[a4paper, margin=1in]{geometry} % 设置A4页面和1英寸边距

% --- 数学公式相关宏包 ---
\usepackage{amsmath}   % 强大的数学公式环境,如 align*
\usepackage{amssymb}   % 提供更多数学符号,如 \mathbb
\usepackage{bm}        % 处理数学公式中的粗体

% --- 其他功能性宏包 ---
\usepackage{enumitem}  % 优化列表格式
\usepackage{hyperref}  % 创建超链接
\hypersetup{
    colorlinks=true,
    linkcolor=blue,
    filecolor=magenta,      
    urlcolor=cyan,
    pdftitle={强化学习笔记},
    pdfpagemode=FullScreen,
}

% --- 自定义数学命令 ---
% 定义 argmax 操作符,使其下标显示在正下方
\DeclareMathOperator*{\argmax}{arg\,max}

% --- 文档开始 ---
\begin{document}

\title{强化学习笔记:价值函数与贝尔曼方程}
\author{} % 作者留空
\date{2025年7月20日}
\maketitle

\section{State Value (状态价值函数)}

\begin{itemize}
    \item \textbf{状态价值(State Value)}:在遵循策略 $\pi$ 的前提下,从状态 $s$ 出发所能获得的期望回报(Expected Return)。
    \item \textbf{回报(Return)}:从某个时刻开始,后续所有奖励的折扣总和,是一条具体轨迹的产出。
    \item \textbf{价值(Value)}:对所有可能轨迹的回报求期望,是一个统计量。
\end{itemize}

\textbf{状态价值函数 (State-Value Function) 定义为:}
\begin{equation*}
v_\pi(s) = \mathbb{E}_\pi \bigl[G_t \mid S_t = s\bigr]
\end{equation*}
其中:
\begin{itemize}
    \item $G_t = \displaystyle\sum_{k=0}^\infty \gamma^k R_{t+k+1}$ 表示从时刻 $t$ 开始的总折扣回报;
    \item $\mathbb{E}_\pi[\cdot]$ 表示在策略 $\pi$ 下的期望。
\end{itemize}

\hrulefill

\section{Bellman Equation (贝尔曼方程)}

\textbf{状态价值函数的贝尔曼展开:}
\begin{equation*}
v_\pi(s) = \mathbb{E}_\pi \bigl[R_{t+1} + \gamma\,G_{t+1} \mid S_t = s\bigr]
\end{equation*}

进一步边缘化动作和后继状态,可得:
\begin{equation*}
v_\pi(s) = \sum_a \pi(a \mid s)\,\Bigl(\mathcal{R}_s^a + \gamma \sum_{s'} \mathcal{P}_{ss'}^a\,v_\pi(s')\Bigr)
\end{equation*}

或者写成期望形式:
\begin{equation*}
v_\pi(s) = \sum_a \pi(a \mid s)\,\mathbb{E}\bigl[R_{t+1} + \gamma\,v_\pi(S_{t+1}) \mid S_t = s, A_t = a\bigr]
\end{equation*}
其中:
\begin{itemize}
    \item $\mathcal{R}_s^a = \mathbb{E}[R_{t+1} \mid S_t = s, A_t = a]$;
    \item $\mathcal{P}_{ss'}^a = p(s' \mid s, a)$。
\end{itemize}

\hrulefill

\section{矩阵形式的贝尔曼方程}

首先,为所有状态 $s \in \mathcal{S}$ 写出:
\begin{equation*}
v_\pi(s) = \sum_a \pi(a \mid s)\,\mathcal{R}_s^a \;+\;\gamma\,\sum_a \pi(a \mid s)\,\sum_{s'}\mathcal{P}_{ss'}^a\,v_\pi(s')
\end{equation*}

定义:
\begin{itemize}
    \item 价值向量 $\mathbf{v}_\pi$,其中第 $s$ 项为 $v_\pi(s)$;
    \item 奖励向量 $\mathbf{r}_\pi$,其中第 $s$ 项为 $r_\pi(s)=\sum_a\pi(a\mid s)\,\mathcal{R}_s^a$;
    \item 状态转移矩阵 $\mathbf{P}_\pi$,其中第 $(s,s')$ 项为 $P_\pi(s,s')=\sum_a\pi(a\mid s)\,\mathcal{P}_{ss'}^a$。
\end{itemize}

于是矩阵形式:
\begin{equation*}
\mathbf{v}_\pi = \mathbf{r}_\pi + \gamma\,\mathbf{P}_\pi\,\mathbf{v}_\pi
\end{equation*}

解得:
\begin{equation*}
\mathbf{v}_\pi = (I - \gamma\,\mathbf{P}_\pi)^{-1}\,\mathbf{r}_\pi
\end{equation*}

\hrulefill

\section{价值函数的迭代求解方法}

\subsection{策略评估 (Policy Evaluation)}
从初始值 $\mathbf{v}^{(0)}$ 开始迭代:
\begin{equation*}
\mathbf{v}^{(k+1)} = \mathbf{r}_\pi + \gamma\,\mathbf{P}_\pi\,\mathbf{v}^{(k)}
\end{equation*}
直到满足:
\begin{equation*}
\|\mathbf{v}^{(k+1)} - \mathbf{v}^{(k)}\|_\infty < \epsilon
\end{equation*}

\subsection{策略迭代 (Policy Iteration)}
\begin{enumerate}
    \item 初始化策略 $\pi_0$;
    \item 使用策略评估计算 $v_{\pi_k}$;
    \item 策略改进:
    \begin{equation*}
    \pi_{k+1}(s) = \argmax_a \Bigl(\mathcal{R}_s^a + \gamma \sum_{s'} \mathcal{P}_{ss'}^a\,v_{\pi_k}(s')\Bigr)
    \end{equation*}
    \item 若 $\pi_{k+1} = \pi_k$ 则停止,否则返回第 2 步。
\end{enumerate}

\subsection{值迭代 (Value Iteration)}
从任意初始值 $v^{(0)}$ 开始迭代:
\begin{equation*}
v^{(k+1)}(s) = \max_a \Bigl(\mathcal{R}_s^a + \gamma \sum_{s'} \mathcal{P}_{ss'}^a\,v^{(k)}(s')\Bigr)
\end{equation*}
当收敛到 $v^*$ 后,可通过
\begin{equation*}
\pi^*(s) = \argmax_a \Bigl(\mathcal{R}_s^a + \gamma \sum_{s'} \mathcal{P}_{ss'}^a\,v^*(s')\Bigr)
\end{equation*}
提取得到最优策略。

\hrulefill

\section{Action Value (动作价值函数)}

\textbf{动作价值函数 (Action-Value Function) 定义为:}
\begin{align*}
q_\pi(s, a) &= \mathbb{E}_\pi \bigl[G_t \mid S_t = s, A_t = a\bigr] \\
            &= \mathcal{R}_s^a + \gamma \sum_{s'} \mathcal{P}_{ss'}^a\,v_\pi(s')
\end{align*}

状态价值与动作价值的关系:
\begin{equation*}
v_\pi(s) = \sum_a \pi(a \mid s)\,q_\pi(s, a)
\end{equation*}

\hrulefill

\section{贝尔曼最优方程 (Bellman Optimality Equation)}

最优状态价值函数:
\begin{equation*}
v^*(s) = \max_a q^*(s, a)
\end{equation*}

最优动作价值函数:
\begin{equation*}
q^*(s, a) = \mathcal{R}_s^a + \gamma \sum_{s'} \mathcal{P}_{ss'}^a\,v^*(s')
\end{equation*}

综合得:
\begin{align*}
v^*(s) &= \max_a \Bigl(\mathcal{R}_s^a + \gamma \sum_{s'} \mathcal{P}_{ss'}^a\,v^*(s')\Bigr) \\
q^*(s, a) &= \mathcal{R}_s^a + \gamma \sum_{s'} \mathcal{P}_{ss'}^a\,\max_{a'} q^*(s', a')
\end{align*}

\hrulefill

\section{不动点定理与收缩映射}

\textbf{收缩映射 (Contraction Mapping) 定义:}

在完备度量空间 $(X,d)$ 中,若映射 $f:X\to X$ 满足
\begin{equation*}
d\bigl(f(x),f(y)\bigr)\le k\,d(x,y),\quad k\in[0,1),
\end{equation*}
则称 $f$ 为收缩映射。
\bigskip

\textbf{巴拿赫不动点定理 (Banach Fixed-Point Theorem):}

任何收缩映射 $f$ 都存在唯一不动点 $x^*$,并且迭代
\begin{equation*}
x_{n+1}=f(x_n)
\end{equation*}
必收敛于 $x^*$。
\bigskip

\textbf{贝尔曼最优算子:}

定义算子 $\mathcal{T}$ 为
\begin{equation*}
(\mathcal{T}v)(s) = \max_a \Bigl(\mathcal{R}_s^a + \gamma \sum_{s'} \mathcal{P}_{ss'}^a\,v(s')\Bigr).
\end{equation*}
该算子是 $\gamma$-收缩,因此存在唯一不动点 $v^*$,即最优价值函数,且值迭代
\begin{equation*}
v^{(k+1)}=\mathcal{T}v^{(k)}
\end{equation*}
必收敛于 $v^*$。

\end{document}