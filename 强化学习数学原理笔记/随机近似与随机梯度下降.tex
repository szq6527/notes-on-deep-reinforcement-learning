\documentclass[12pt, a4paper]{article}

% --- 包的设置 (Preamble) ---
\usepackage[UTF8]{ctex} % 中文支持
\usepackage{amsmath, amssymb, amsfonts} % AMS数学公式包
\usepackage{geometry} % 页面布局
\usepackage{booktabs} % 专业表格
\usepackage{enumitem} % 自定义列表
\usepackage{hyperref} % 超链接
\usepackage{xcolor} % 颜色
\usepackage[most]{tcolorbox} % 用于创建带颜色的文本框

% --- 页面布局设置 ---
\geometry{a4paper, left=2.5cm, right=2.5cm, top=2.5cm, bottom=2.5cm}

% --- 超链接设置 ---
\hypersetup{
    colorlinks=true,
    linkcolor=blue,
    filecolor=magenta,      
    urlcolor=cyan,
    pdftitle={Robbins-Monro 算法},
    pdfpagemode=FullScreen,
    pdfauthor={szq},
}

% --- 自定义命令 ---
\newcommand{\E}{\mathbb{E}} % 期望符号

% --- 定理环境设置 (使用 tcolorbox) ---
\newtcbtheorem{theorembox}{定理}% The displayed name
  {colback=yellow!25!white, colframe=black!75!white, fonttitle=\bfseries, arc=1mm}
  {thm}

% --- 文档标题 ---
\title{\textbf{Robbins-Monro (RM) 算法学习笔记}}
\author{szq (根据PPT内容整理)}
\date{\today}

% --- 文档开始 ---
\begin{document}

\maketitle
\tableofcontents
\newpage

\section{问题陈述:寻找函数根}

假设我们的目标是找到以下方程的根:
\[
g(w) = 0,
\]
其中 $w \in \mathbb{R}$ 是待求解的变量,而 $g: \mathbb{R} \to \mathbb{R}$ 是一个函数。

\begin{itemize}
    \item 许多问题最终都可以转化为这种寻根问题。例如,假设 $J(w)$ 是一个需要最小化的目标函数,那么这个优化问题可以转化为求解:
    \[
    g(w) = \nabla_w J(w) = 0
    \]
    
    \item 另外,像 $g(w) = c$(其中 $c$ 是一个常数)这样的方程也可以通过将其重写为一个新函数 $g(w) - c = 0$ 来转化为上述标准形式。
\end{itemize}

\section{Robbins-Monro (RM) 算法}

\subsection{算法提出的背景}

\textbf{如何求解 $g(w) = 0$ 的根?}

\begin{itemize}
    \item 如果函数 $g$ 或其导数的\textbf{表达式已知},那么有许多成熟的数值算法可以解决这个问题。
    
    \item \textbf{核心问题:} 如果函数 $g$ 的\textbf{表达式未知},该怎么办?例如,函数 $g$ 可能由一个我们无法获得其精确表达式的人工神经网络表示。
\end{itemize}
在这种场景下,我们无法直接计算 $g(w)$ 的值,只能通过某种方式(例如,与环境交互)获得其\textbf{带有噪声的观测值}。

\subsection{算法描述}
Robbins-Monro (RM) 算法正是为解决这类问题而设计的。其迭代更新公式如下:
\[
w_{k+1} = w_k - a_k \tilde{g}(w_k, \eta_k), \quad k = 1, 2, 3, \dots
\]
其中:
\begin{itemize}
    \item $w_k$ 是对根的第 $k$ 次估计。
    \item $\tilde{g}(w_k, \eta_k) = g(w_k) + \eta_k$ 是对 $g(w_k)$ 的第 $k$ 次\textbf{带噪观测},$\eta_k$ 是随机噪声。
    \item $a_k$ 是一个正系数,通常称为步长或学习率。
\end{itemize}

函数 $g(w)$ 在这里是一个\textbf{黑箱 (black box)}!该算法依赖于数据:
\begin{itemize}
    \item \textbf{输入序列:} $\{w_k\}$
    \item \textbf{带噪输出序列:} $\{\tilde{g}(w_k, \eta_k)\}$
\end{itemize}
\textcolor{blue}{\textbf{核心思想:}} 当模型(即函数的精确表达式)未知时,我们必须依赖\textbf{数据}。

\subsection{收敛性分析}
为什么 RM 算法能够找到 $g(w)=0$ 的根?通常的解释分为两步:
\begin{enumerate}
    \item 首先通过一个直观的例子来展示其工作原理。
    \item 其次给出严格的收敛性分析。
\end{enumerate}
下面的定理给出了 RM 算法收敛的严格数学证明。

\begin{theorembox}{Robbins-Monro Theorem}{rm_theorem}
在 Robbins-Monro 算法中,如果满足以下条件:
\begin{enumerate}[label=\arabic*)]
    \item 函数 $g(w)$ 的梯度(斜率)被正数界定:
    \[ 0 < c_1 \le \nabla_w g(w) \le c_2 \quad \text{for all } w; \]
    
    \item 步长序列 $a_k$ 满足:
    \[ \sum_{k=1}^{\infty} a_k = \infty \quad \text{并且} \quad \sum_{k=1}^{\infty} a_k^2 < \infty; \]
    
    \item 噪声 $\eta_k$ 在给定历史信息 $\mathcal{H}_k$ 的条件下,条件期望为零,且条件二阶矩有界:
    \[ \E[\eta_k | \mathcal{H}_k] = 0 \quad \text{并且} \quad \E[\eta_k^2 | \mathcal{H}_k] < \infty; \]
    其中 $\mathcal{H}_k = \{w_k, w_{k-1}, \dots\}$ 代表到第 $k$ 步为止的历史信息。
\end{enumerate}
那么,序列 $w_k$ 将以概率 1 (with probability 1, w.p.1) 收敛到方程的根 $w^*$,满足 $g(w^*) = 0$。
\end{theorembox}

\section{收敛条件详解}
下面我们来详细解释 Robbins-Monro 定理中的三个核心条件。

\subsection{条件一:关于函数 $g(w)$}
\[ 
0 < c_1 \le \nabla_w g(w) \le c_2 \quad \text{for all } w 
\]
这个条件表明:
\begin{itemize}
    \item 函数 $g(w)$ 是\textbf{单调递增}的,这确保了方程 $g(w)=0$ 的根存在且\textbf{唯一}。
    \item 函数的梯度(斜率)有一个上界,这可以防止函数值变化过快,有助于算法的稳定。
\end{itemize}

\subsection{条件二:关于步长序列 $a_k$}
\[ 
\sum_{k=1}^{\infty} a_k^2 < \infty \quad \text{且} \quad \sum_{k=1}^{\infty} a_k = \infty 
\]
这个对步长的双重条件是收敛的关键,可以分为两部分来理解:

\paragraph{第一部分:$\sum_{k=1}^{\infty} a_k^2 < \infty$ (保证收敛到稳定点)}
这个条件(级数收敛)保证了步长 $a_k$ 最终会趋向于零,即 $\lim_{k \to \infty} a_k = 0$。
\begin{itemize}
    \item \textbf{重要性:} 考虑更新公式 $w_{k+1} - w_k = -a_k \tilde{g}(w_k, \eta_k)$。为了让序列 $w_k$ 最终收敛,其更新步长 $w_{k+1} - w_k$ 必须趋于 0。当 $w_k$ 接近根 $w^*$ 时,$g(w_k)$ 会趋于 0,但噪声项 $\eta_k$ 不会。因此,必须通过让 $a_k \to 0$ 来抑制噪声的持续影响,确保更新量最终消失,使得算法稳定在根附近。
\end{itemize}

\paragraph{第二部分:$\sum_{k=1}^{\infty} a_k = \infty$ (保证能到达根)}
这个条件(级数发散)保证了步长 $a_k$ \textbf{收敛到零的速度不能太快}。
\begin{itemize}
    \item \textbf{重要性:} 将更新规则从 $k=1$ 开始累加,形式上可以理解为:
    \[ 
    w_{\infty} - w_1 = \sum_{k=1}^{\infty} (w_{k+1} - w_k) = - \sum_{k=1}^{\infty} a_k \tilde{g}(w_k, \eta_k) 
    \]
    如果步长序列的和 $\sum a_k$ 是一个有限值,那么总的“探索距离”也将是有限的。这意味着,如果初始猜测值 $w_1$ 离真正的根 $w^*$ 非常远,算法可能永远无法到达它。而“和发散”的条件确保了算法有能力跨越任意有限的距离,不会因为步长衰减过快而“半途而废”。
\end{itemize}

\subsection{条件三:关于噪声 $\eta_k$}
\[ 
\E[\eta_k | \mathcal{H}_k] = 0 \quad \text{并且} \quad \E[\eta_k^2 | \mathcal{H}_k] < \infty 
\]
这个条件要求噪声在给定历史信息的条件下,其\textbf{条件期望为零},且条件二阶矩有界(即方差有界)。
\begin{itemize}
    \item \textbf{条件期望为零}意味着,平均而言,噪声不会系统性地将估计值推向某个错误的方向,它是无偏的。
    \item \textbf{方差有界}确保了噪声不会出现极端离群值,从而破坏收敛进程。
    \item 一个常见但更强的特例是,噪声序列 $\{\eta_k\}$ 是一个独立同分布 (i.i.d.) 的随机序列,满足 $\E[\eta_k] = 0$ 和 $\E[\eta_k^2] < \infty$。
    \item 需要注意的是,该定理并不要求噪声服从高斯分布 (Gaussian)。
\end{itemize}

\section{随机梯度下降 (Stochastic Gradient Descent)}
接下来,我们介绍随机梯度下降(SGD)算法。
\begin{itemize}
    \item SGD 算法在机器学习和强化学习领域被广泛使用。
    \item SGD 是一种特殊的 RM 算法。
    \item (之前讨论的)均值估计算法也是一种特殊的 SGD 算法。
\end{itemize}
假设我们的目标是解决以下优化问题:
\[
\min_w J(w) = \E_X[f(w, X)]
\]
这是一个目标函数包含对随机变量求期望的优化问题。
\begin{itemize}
    \item $w$ 是需要优化的参数。
    \item $X$ 是一个随机变量,期望是针对 $X$ 的分布计算的。
    \item $w$ 和 $X$ 既可以是标量也可以是向量,而函数 $f(\cdot)$ 的输出是标量。
\end{itemize}

为了解决这个优化问题,有几种不同的梯度下降方法。

\subsection{方法一:梯度下降 (Gradient Descent, GD)}
标准的梯度下降法使用目标函数对参数 $w$ 的\textbf{真实梯度}进行更新。
\[
w_{k+1} = w_k - \alpha_k \nabla_w \E_X[f(w_k, X)] = w_k - \alpha_k \E_X[\nabla_w f(w_k, X)]
\]
\paragraph{缺点 (Drawback):} 真实梯度的期望值 $\E_X[\nabla_w f(w_k, X)]$ 通常难以直接计算,因为它可能需要对 $X$ 的整个概率分布进行积分。

\subsection{方法二:批量梯度下降 (Batch Gradient Descent, BGD)}
批量梯度下降通过在一批样本上计算梯度的平均值来\textbf{近似}真实的期望梯度。
\[
\E_X[\nabla_w f(w_k, X)] \approx \frac{1}{n} \sum_{i=1}^{n} \nabla_w f(w_k, x_i)
\]
其更新规则为:
\[
w_{k+1} = w_k - \alpha_k \frac{1}{n} \sum_{i=1}^{n} \nabla_w f(w_k, x_i)
\]
\paragraph{缺点 (Drawback):} 在每一次迭代更新 $w_k$ 时,都需要采集并计算大量的样本(一个 batch),计算成本可能很高。

\subsection{方法三:随机梯度下降 (Stochastic Gradient Descent, SGD)}
随机梯度下降是 GD 和 BGD 的一种简化,它在每次迭代时仅使用\textbf{单个}随机样本来估计梯度。
\[
w_{k+1} = w_k - \alpha_k \nabla_w f(w_k, x_k)
\]
其中 $x_k$ 是在第 $k$ 步从 $X$ 的分布中随机采样的一个样本。

\begin{itemize}
    \item \textbf{与梯度下降 (GD) 的比较:} SGD 用\textbf{随机梯度} (stochastic gradient) $\nabla_w f(w_k, x_k)$ 替代了\textbf{真实梯度} (true gradient) $\E_X[\nabla_w f(w_k, X)]$。这个随机梯度是真实梯度的一个\textbf{无偏估计}。
    \item \textbf{与批量梯度下降 (BGD) 的比较:} SGD 相当于将批量大小设置为 $n=1$ 的 BGD,极大地降低了单次迭代的计算复杂度。
\end{itemize}



\end{document}