\documentclass[12pt, a4paper]{article}

% --- 包的设置 (Preamble) ---
\usepackage[UTF8]{ctex} % 中文支持
\usepackage{amsmath, amssymb, amsfonts} % AMS数学公式包
\usepackage{geometry} % 页面布局
\usepackage{graphicx} % 插入图片
\usepackage{booktabs} % 专业表格
\usepackage{enumitem} % 自定义列表
\usepackage{hyperref} % 超链接
\usepackage{xcolor} % 颜色

% --- 页面布局设置 ---
\geometry{a4paper, left=2.5cm, right=2.5cm, top=2.5cm, bottom=2.5cm}

% --- 超链接设置 ---
\hypersetup{
    colorlinks=true,
    linkcolor=blue,
    filecolor=magenta,      
    urlcolor=cyan,
    pdftitle={强化学习算法解读},
    pdfpagemode=FullScreen,
}

% --- 自定义数学命令 ---
\newcommand{\argmax}{\operatornamewithlimits{argmax}} % argmax
\newcommand{\vect}[1]{\mathbf{#1}} % 向量加粗

% --- 文档标题 ---
\title{\textbf{强化学习经典算法解读:价值迭代与策略迭代}}
\author{szq (根据PPT内容整理)}
\date{2025年7月26日}

% --- 文档开始 ---
\begin{document}

\maketitle
\tableofcontents
\newpage

\section{引言}
本文档根据提供的PPT内容,详细解读了强化学习中两种经典的动态规划(Dynamic Programming, DP)算法:\textbf{价值迭代(Value Iteration)}和\textbf{策略迭代(Policy Iteration)}。这两种算法都是在已知环境模型(即状态转移概率 $P$ 和奖励 $R$)的前提下,求解马尔可夫决策过程(MDP)最优策略和最优价值函数的核心方法。

\section{价值迭代 (Value Iteration)}

\subsection{核心思想与算法引出}
价值迭代的目标是求解\textbf{贝尔曼最优方程 (Bellman Optimality Equation)}:
\[
v^*(s) = \max_{a \in \mathcal{A}} \sum_{s', r} p(s', r | s, a) [r + \gamma v^*(s')]
\]
在向量形式下,可以写作:
\[
\vect{v} = \max_{\pi} (\vect{r}_{\pi} + \gamma \vect{P}_{\pi} \vect{v})
\]
由于该方程中存在 $\max$ 算子,导致其为非线性方程组,无法直接求解。

根据\textbf{收缩映射定理 (Contraction Mapping Theorem)},我们可以将贝尔曼最优方程的右侧视为一个更新算子,通过反复迭代来逼近最优解。这便引出了价值迭代算法。

\subsection{算法描述}
价值迭代从一个任意的初始价值函数 $\vect{v}_0$ 开始,通过以下公式进行迭代更新:
\[
\vect{v}_{k+1} = \max_{\pi} (\vect{r}_{\pi} + \gamma \vect{P}_{\pi} \vect{v}_k), \quad k=0, 1, 2, \dots
\]
这个过程会持续进行,直到价值函数的变化足够小(即 $\lVert \vect{v}_{k+1} - \vect{v}_k \rVert < \epsilon$),此时的 $\vect{v}_{k+1}$ 就近似为最优价值函数 $\vect{v}^*$。

\subsection{算法分解与元素形式}
价值迭代的每一步更新 `v_{k+1}(s) = ...` 都可以被逻辑上分解为两个步骤,这有助于理解其计算过程。

\begin{enumerate}[label=\textbf{Step \arabic*}:, wide, labelwidth=!, labelindent=0pt]
    \item \textbf{策略更新 (Policy Update) (隐式)} \\
    在这一步,我们基于当前的价值估计 $\vect{v}_k$,找到一个临时的贪心策略 $\pi_{k+1}$。对于每一个状态 $s \in \mathcal{S}$,该策略选择能够最大化期望回报的动作:
    \[
    \pi_{k+1}(s) = \argmax_{a \in \mathcal{A}} \left\{ \sum_{s', r} p(s', r | s, a) [r + \gamma v_k(s')] \right\}
    \]
    我们通常将括号内的部分定义为动作价值函数 $q_k(s, a)$:
    \[
    q_k(s, a) = \sum_{s', r} p(s', r | s, a) [r + \gamma v_k(s')]
    \]
    因此,策略更新可以简化为:$\pi_{k+1}(s) = \argmax_{a} q_k(s, a)$。

    \item \textbf{价值更新 (Value Update)} \\
    接着,我们使用上一步找到的贪心策略来更新状态的价值。由于该策略总是选择最优动作,所以新的价值就是该状态下所有动作价值中的最大值:
    \[
    v_{k+1}(s) = \max_{a \in \mathcal{A}} q_k(s, a)
    \]
\end{enumerate}

\textbf{重要说明:} 在迭代过程中(收敛前),$\vect{v}_k$ 并不是任何一个固定策略下的真实价值函数。它仅仅是通往最优价值函数 $\vect{v}^*$ 的一个中间估计值。


\section{策略迭代 (Policy Iteration)}
策略迭代是另一种求解MDP的经典算法。与价值迭代直接优化价值函数不同,策略迭代在\textbf{策略空间}中进行搜索,通过交替评估和改进策略来找到最优策略。

\subsection{算法描述}
算法从一个随机的初始策略 $\pi_0$ 开始,循环执行以下两个步骤,直到策略不再发生变化。
\begin{enumerate}[label=\textbf{Step \arabic*}:, wide, labelwidth=!, labelindent=0pt]
    \item \textbf{策略评估 (Policy Evaluation, PE)} \\
    对于当前固定的策略 $\pi_k$,精确地计算其对应的状态价值函数 $\vect{v}_{\pi_k}$。这需要求解\textbf{贝尔曼期望方程}:
    \[
    \vect{v}_{\pi_k} = \vect{r}_{\pi_k} + \gamma \vect{P}_{\pi_k} \vect{v}_{\pi_k}
    \]
    这是一个线性方程组。对于大型问题,通常使用迭代法求解:
    \[
    \vect{v}_{\pi_k}^{(j+1)} = \vect{r}_{\pi_k} + \gamma \vect{P}_{\pi_k} \vect{v}_{\pi_k}^{(j)}
    \]
    这个内部迭代会进行直到 $\vect{v}_{\pi_k}^{(j)}$ 收敛,得到精确的 $\vect{v}_{\pi_k}$。

    \item \textbf{策略改进 (Policy Improvement, PI)} \\
    利用上一步计算出的精确价值函数 $\vect{v}_{\pi_k}$,找到一个更好的新策略 $\pi_{k+1}$。这个新策略对于每个状态 $s$ 都是贪心的:
    \[
    \pi_{k+1}(s) = \argmax_{a \in \mathcal{A}} \left\{ \sum_{s', r} p(s', r | s, a) [r + \gamma v_{\pi_k}(s')] \right\}
    \]
    根据\textbf{策略改进定理},这样得到的新策略 $\pi_{k+1}$ 总是优于或等于旧策略 $\pi_k$。
\end{enumerate}
这个过程 `$\pi_0 \xrightarrow{PE} \vect{v}_{\pi_0} \xrightarrow{PI} \pi_1 \xrightarrow{PE} \vect{v}_{\pi_1} \xrightarrow{PI} \dots$` 会持续进行,直到 $\pi_{k+1} = \pi_k$。此时,算法收敛,找到了最优策略 $\pi^*$ 和最优价值函数 $\vect{v}^*$。

\section{总结与对比}
价值迭代和策略迭代是解决MDP问题的两种 foundational 算法。它们都保证收敛到最优解,但在计算过程和效率上有所不同。

\begin{table}[h!]
\centering
\caption{价值迭代与策略迭代的对比}
\label{tab:comparison}
\begin{tabular}{lll}
\toprule
\textbf{特性} & \textbf{价值迭代 (Value Iteration)} & \textbf{策略迭代 (Policy Iteration)} \\
\midrule
\textbf{迭代对象} & 价值函数 $\vect{v}_k$ & 策略 $\pi_k$ \\
\addlinespace
\textbf{核心操作} & 一步完成价值更新和隐式策略改进。 & 分为两步:策略评估和策略改进。 \\
& $\vect{v}_{k+1} = \max_a Q(s,a; \vect{v}_k)$ & 1. PE: 求解线性方程组 (开销大)。\\
& & 2. PI: 贪心更新策略 (开销小)。\\
\addlinespace
\textbf{每次迭代} & 计算简单,只需遍历所有状态和动作。 & 计算复杂,PE步骤本身需要一个\\
\textbf{的复杂度} & & 完整的迭代过程直至收敛。 \\
\addlinespace
\textbf{收敛速度} & 通常需要较多次迭代才能收敛。 & 外部循环(策略改进)的迭代次数\\
& & 通常很少,能很快找到最优策略。\\
\addlinespace
\textbf{联系} & 可看作“截断的”策略迭代,即PE & 每次改进策略前,都要求对当前\\
& 步骤只迭代一次。 & 策略的价值进行充分评估。\\
\bottomrule
\end{tabular}
\end{table}

理解这两种算法的原理、区别与联系,对于学习后续更高级的强化学习算法,如Q-Learning和Actor-Critic等,至关重要。

\end{document}