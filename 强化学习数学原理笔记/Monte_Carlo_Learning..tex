\documentclass[12pt, a4paper]{article}

% --- 包的设置 (Preamble) ---
\usepackage[UTF8]{ctex} % 中文支持
\usepackage{amsmath, amssymb, amsfonts} % AMS数学公式包
\usepackage{geometry} % 页面布局
\usepackage{booktabs} % 专业表格
\usepackage{enumitem} % 自定义列表
\usepackage{hyperref} % 超链接 (它提供了 \texorpdfstring 命令)
\usepackage{xcolor} % 颜色

% --- 页面布局设置 ---
\geometry{a4paper, left=2.5cm, right=2.5cm, top=2.5cm, bottom=2.5cm}

% --- 超链接设置 ---
\hypersetup{
    colorlinks=true,
    linkcolor=blue,
    filecolor=magenta,      
    urlcolor=cyan,
    pdftitle={强化学习算法解读:蒙特卡洛方法},
    pdfpagemode=FullScreen,
    pdfauthor={szq},
}

% --- 自定义数学命令 ---
\newcommand{\argmax}{\operatornamewithlimits{argmax}} % argmax (专业定义)
\newcommand{\E}{\mathbb{E}} % 期望符号

% --- 文档标题 ---
\title{\textbf{强化学习经典算法解读:蒙特卡洛方法}}
\author{szq (根据PPT内容整理)}
\date{2025年7月27日}

% --- 文档开始 ---
\begin{document}

\maketitle
\tableofcontents
\newpage

\section{蒙特卡洛(Monte Carlo)方法介绍}
根据课程的规划,我们将逐步学习以下几种基于蒙特卡洛(MC)的强化学习算法:
\begin{enumerate}
    \item 启发性例子 (Motivating example)
    \item 最简单的基于 MC 的强化学习算法 (MC Basic)
    \item 更高效地利用数据 (MC Exploring Starts)
    \item 无需“探索性开端”的 MC 算法 (MC $\varepsilon$-Greedy)
\end{enumerate}
本文档将按照此顺序,逐步深入解读这些算法。

\subsection{核心思想:从“需要模型”到“模型无关”}
在强化学习中,我们的目标是评估一个策略的好坏,通常通过计算其价值函数。动作价值函数 $q_\pi(s, a)$ 有两种主要的表达形式。
\begin{itemize}
    \item \textbf{表达式一(需要模型):} 依赖于环境的动态模型 $p(s'|s,a)$ 和奖励模型 $p(r|s,a)$。
    \[
        q_{\pi_k}(s, a) = \sum_{r} p(r|s, a)r + \gamma \sum_{s'} p(s'|s, a)v_{\pi_k}(s')
    \]
    
    \item \textbf{表达式二(无需模型):} 将动作价值定义为回报(Return)的期望。
    \[
        q_{\pi_k}(s, a) = \E[G_t | S_t = s, A_t = a]
    \]
\end{itemize}
\textcolor{blue}{\textbf{实现模型无关(Model-Free)RL 的核心思想:}} 我们可以利用表达式二,直接基于与环境交互产生的\textbf{数据}(样本或经验),来估计动作价值函数 $q_{\pi_k}(s, a)$。

\subsection{动作价值的蒙特卡洛估计流程}
蒙特卡洛方法正是利用大数定律,通过采样来估计期望值。
\begin{itemize}
    \item 从一个指定的“状态-动作”对 $(s, a)$ 出发,遵循当前策略 $\pi_k$ 生成一个完整的幕(episode)。
    \item 这个幕的回报 $g(s,a)$ 就是总回报 $G_t$ 的一个\textbf{样本}。
    \item 通过生成 $N$ 个幕并取回报的平均值,我们可以近似期望值:
    \[
        q_{\pi_k}(s, a) = \E[G_t|S_t=s, A_t=a] \approx \frac{1}{N} \sum_{i=1}^{N} g^{(i)}(s, a)
    \]
\end{itemize}
\textcolor{red}{\textbf{基本思想:}}当模型未知时,我们可以用\textbf{数据(data)}进行估计。


\section{最简单的 MC 强化学习算法 (MC Basic)}
\subsection{算法描述}
给定初始策略 $\pi_0$,在第 $k$ 次迭代中,算法包括两个步骤:
\begin{enumerate}[label=\textbf{Step \arabic*}:, wide, labelwidth=!, labelindent=0pt]
    \item \textbf{策略评估 (Policy Evaluation)}: 对每个 $(s,a)$,从其出发运行足够多的幕,用回报的平均值来近似 $q_{\pi_k}(s, a)$。
    \item \textbf{策略改进 (Policy Improvement)}: 使用贪心策略来改进策略,$\pi_{k+1}(s) = \argmax_{a} q_{\pi_k}(s, a)$。
\end{enumerate}

\subsection{与策略迭代(Policy Iteration)的对比}
MC Basic 算法的流程与\textbf{策略迭代算法}几乎完全相同,区别在于\textbf{策略评估}:MC Basic 通过\textbf{采样求平均}来估计 $q_{\pi_k}(s, a)$,而非求解贝尔曼方程。


\section{提高数据利用效率:First-Visit 与 Every-Visit MC}
MC Basic 算法为了评估 $q(s,a)$,需要从该 $(s,a)$ 出发生成大量幕,数据效率低下。一个幕实际上可以用来更新其中所有出现过的状态-动作对。

\subsection{数据利用问题与改进思路}
考虑一个幕:$s_1 \xrightarrow{a_2} s_2 \xrightarrow{a_4} s_1 \xrightarrow{a_2} s_2 \xrightarrow{a_3} s_5 \to \dots$
\begin{itemize}
    \item \textbf{MC Basic 的做法 (Initial-visit method)}: 只用这个幕来评估 $q_\pi(s_1, a_2)$。
    \item \textbf{缺点}: 浪费了幕中包含的 $(s_2, a_4), (s_2, a_3)$ 等其他信息。
\end{itemize}

\subsection{更高效的数据利用方法}
当一个 $(s,a)$ 在一个幕中被访问时,我们可以利用其后续的回报来更新 $q(s,a)$。当 $(s,a)$ 被多次访问时,有两种主流方法:
\begin{itemize}
    \item \textbf{首次访问蒙特卡洛法 (First-visit MC)}: 仅使用该幕中 $(s,a)$ \textbf{第一次}出现时所得到的回报来更新。
    \item \textbf{每次访问蒙特卡洛法 (Every-visit MC)}: 使用该幕中 $(s,a)$ \textbf{每一次}出现时所得到的回报来更新。
\end{itemize}

\section{MC Exploring Starts 算法}
\subsection{探索性开端 (Exploring Starts) 的概念与挑战}
\textbf{问题:为什么需要“探索性开端”?}
\begin{itemize}
    \item \textbf{理论上的必要性}: 为了找到最优动作,必须对每个状态下的\textbf{所有}动作的价值都有准确估计。如果某个动作从未被探索,就可能错过最优解。
\end{itemize}
为此,我们引入\textbf{探索性开端的假设}:假设任意 $(s,a)$ 都有非零概率成为幕的起点。
\paragraph{实践中的挑战:} 这个假设在物理世界等很多应用中难以满足,造成了理论与实践的鸿沟。

\section{无需探索性开端的MC算法:MC \texorpdfstring{$\varepsilon$}{epsilon}-Greedy}
为了移除“探索性开端”的强假设,我们引入“软策略”的概念。

\subsection{软策略 (Soft Policies) 的引入}
\begin{itemize}
    \item \textbf{定义}: 一个策略 $\pi$ 如果对任意状态下的任意动作,选择的概率都为正(即 $\pi(a|s) > 0$ for all $s,a$),那么它就被称为\textbf{软策略 (soft policy)}。
    \item \textbf{引入原因}: 使用软策略,我们只需生成足够长的幕,就有机会访问到每一个状态-动作对。这样就不再需要强制让幕从每一个状态-动作对开始。
\end{itemize}

\subsection{\texorpdfstring{$\varepsilon$}{epsilon}-贪心策略 (\texorpdfstring{$\varepsilon$}{epsilon}-Greedy Policies)}
我们将使用的具体软策略是 \textbf{$\varepsilon$-贪心策略}。
\[
\pi(a|s) = 
\begin{cases}
    1 - \varepsilon + \frac{\varepsilon}{|\mathcal{A}(s)|} & \text{对于贪心动作 (greedy action)} \\
    \frac{\varepsilon}{|\mathcal{A}(s)|} & \text{对于其他 } |\mathcal{A}(s)|-1 \text{ 个动作}
\end{cases}
\]
其中 $\varepsilon \in [0, 1]$。它在\textbf{探索 (exploration)} 和\textbf{利用 (exploitation)} 之间取得了平衡。

\subsection{MC \texorpdfstring{$\varepsilon$}{epsilon}-Greedy 算法描述}
我们将 $\varepsilon$-贪心策略嵌入到算法中,通过修改\textbf{策略改进}步骤。
\paragraph{原始的策略改进(确定性)}
$\pi_{k+1}(a|s) = 1$ if $a = \argmax_{a'} q_{\pi_k}(s, a')$, and $0$ otherwise.
\paragraph{新的策略改进($\varepsilon$-Greedy)}
新的策略 $\pi_{k+1}$ 对于当前的价值函数 $q_{\pi_k}$ 是 $\varepsilon$-贪心的:
\[
\pi_{k+1}(a|s) = 
\begin{cases}
    1 - \varepsilon + \frac{\varepsilon}{|\mathcal{A}(s)|}, & \text{if } a = a_k^* \\
    \frac{\varepsilon}{|\mathcal{A}(s)|}, & \text{if } a \neq a_k^*
\end{cases}
\quad \text{其中 } a_k^* = \argmax_{a} q_{\pi_k}(s, a)
\]
\paragraph{算法总结}
\begin{itemize}
    \item \textbf{MC $\varepsilon$-Greedy 算法}与 MC Exploring Starts 算法基本相同,区别在于策略改进步骤使用了 $\varepsilon$-贪心策略。
    \item 它\textbf{不再需要探索性开端}的假设,但仍通过软策略的随机性来保证对所有状态-动作对的持续探索。
\end{itemize}

\end{document}